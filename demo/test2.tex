% nwsuafexam文档类各个选项的含义:
% answers     是否显示答案(缺省为不显示)
% sealed      是否为密封试卷(缺省为普通)
% contitemcnt 是否为各小题题目进行连续编号(缺省为不连续编号)
% prescorebox 显示前置评分框(缺省为不显示)。
% cancelspace 忽略答题空白(在answers下此项无效)
% Tailscore   分值显示位置(题目开头或者题目右边界,缺省为题目开头)
\documentclass[Tailscore,contitemcnt,answers]{nwsuafexam}%当调用或者取消一些选项参数时可能会造成试卷总页数改变,因此建议在调用或取消sealed、cancelspace、answers三个参数中的一个或几个时,先执行“工具”->“清理辅助文件”操作后再编译,否则会造成页数错乱。
\begin{document}	
	% 设置试卷基本信息
	\papertitle{西北农林科技大学本科课程考试试卷}
	\studyyear{2013}{2014}
	\semester{2}
	\subject{农林气象学}
	\testmethod{闭}
	\proteacher{张XX}
	\checkteacher{李XX}
	\papercategory{B}
	% 生成标题
	\maketitle
	%注意事项(提供两个环境参数,分别为总分数和考试时间,环境中默认已经定义了“本试卷共XX道试题, 满分XX分,考试时间XX分钟.”格式的一条注意事项,其余事项可在环境中使用\item命令输入)
	\begin{notice}{100}{120}
		\item 学生在答题前请先填写专业、学号、学院、姓名等基本信息。
	\end{notice}
	
	%综合得分表
	{\heiti
	\begin{tabularx}{\textwidth}{|*{8}{>{\centering\arraybackslash}X|}}
		\hline
		题号 & 一 & 二 & 三 & 四 & 五 & 总分 & 审核人\\
		\hline
		得分 &    &    &    &    &    &     &      \\
		\hline
	\end{tabularx}}
	\begin{questions}
		%填空题       
		\MainQuestion{填空题}{每小题2分,共20分}

		\question 
		气象学中太阳辐射称作短波辐射,把地面和大气辐射称作\blank{长波}辐射。
		
		\question
		对流层的最下层(0-2km)称摩擦层 ,其以上的大气层称\blank{自由大气}。
		
		\question
		大气透明系数与大气的混浊度和入射光的波长有关,波长越短大气透明数越\blank{小};一天中午后的大气透明系数通常最\blank{小}。
		
		\question
		海陆风和疾风的共同特点是形成原因相同,不同点是\blank{周期和范围}。
		
		\question
		大气中的水汽达到饱和通常通过\blank{降温}来实现,当水汽达到饱和时,则相对湿度为\blank{100\%}。
		
		\question
		一天中的最高气温与最低气温之差称为日较差,其值最大的地区出现在\blank{副热带}。
		
		\question
		杨凌$(34~20')$在冬至日正午时刻的太阳高度角为\blank{$32.2^\circ$}度。
		
		\question
		我国季风气候的优点为\blank{雨热同季},缺点为旱涝频繁。
		
		\question
		看中央太天气预报时,预报员对图讲解的部分称作天气形势预报,随后各城市的气温、风速等的预报称作\blank{气象要素}预报。
		
		\question
		某日某时刻某地距地面0.5m、2.0m高度处的气温分别为0.0~$^{\circ}$C和0.5~$^{\circ}$C,则该大气层的稳定度为\blank{绝对稳定},若此地有污染源存在,则该时刻该地的污染将会加重。
		
		%单项选择题
		\MainQuestion{单项选择题}{从下列个题备选答案中选出一个正确答案,并将其代号填写在试卷前面的括号内,答案选错或未选者,该题不得分,每小题1分,共20分}
		\question 
		北京时是\selectline。\selectbracket{B}
		\Choices{北京地方时}{东经$120^\circ$的地方时}{东经$116~20'$的地方时}{东八区的时间}
		
	    \question
	    在对流层中,气温随高度的增加而\selectline。\selectbracket{C}
		\Choices{增加。}{不变。}{降低。}{有时增加有时降低。}
		
		\question
		形成云的基本原因是\selectline。\selectbracket{C}
		\Choices{空气中水汽达到饱和或过饱和。}{空气中有足够的凝结核。}{空气的上升运动。}{空气的下沉运动。}
		
		\question
	    5月1日的赤纬是15N,试问当天在\selectline N以北的地区会出现极昼现象。\selectbracket{B}
	    \Choices{63.5}{75}{78.5}{90}
	    
	    \question
	    晴天水平地面上太阳直接辐射辐照度取决于太阳高度和\selectline。\selectbracket{A}
	    \Choices{大气透明度}{太阳赤纬}{天气条件}{辐射条件}
	    
	    \question
	    暖湿带一般指纬度在\selectline 之间的地带。\selectbracket{C}
	    \Choices{$10^\circ$--$23.5^\circ$}{$23.5^\circ$--$33^\circ$}{$33^\circ$--$45^\circ$}{$45^\circ$--$50^\circ$}
	    
	    \question
	    在赤道处\selectline 为零。\selectbracket{B}
	    \Choices{气压梯度力}{地转偏向力}{惯性离心力}{摩擦力}
	    
	    \question
	    分子散射对太阳光中\selectline 有较明显地减弱作用。\selectbracket{A}
	    \Choices{波长较短部分}{波长较长部分}{中长波部分}{各个波段}
		
		\question
		日平均气温$\geqslant$0$^\circ$C的持续日期成为农事活动上的\selectline。\selectbracket{B}
		\Choices{休闲期}{农耕期}{作物积极生长期}{土壤解冻期}
		
		\question
	    气温\selectline 时,太阳光的有效辐射才可能被绿色植物所利用。\selectbracket{D}
		\Choices{$\geqslant$1$^\circ$C}{$\geqslant$2$^\circ$C}{$\geqslant$ 4$^\circ$C}{$\geqslant$5$^\circ$C}
		
		\question
		0--2m的气层称为\selectline,是农业小气候研究的主要层次。\selectbracket{C}
		\Choices{摩擦层}{对流层}{贴地气层}{近地面层}
		
		\question
		已知5cm深度的土壤温度日振幅为8$^\circ$C,最高温度出现在15h。则该深度土壤温度日较差为\selectline $^\circ$C。\selectbracket{C}
		\Choices{4}{8}{16}{无法确定}
		
		\question
		我国温度的日较差和年较差随着纬度的升高是\selectline。\selectbracket{C}
		\Choices{日较差、年较差均减小}{日较差、年较差均增大}{日较差减小、年较差增大}{日较差增大、年较差减小}
		
		\question
		某时刻气温的铅直分布是随着高度的增加先增加在减小,此气温分布型属于\selectbracket{D}
		\Choices{清晨过渡型}{日射型}{辐射型}{傍晚过渡型}
		
		\question
		视觉热直减率为\selectline。\selectbracket{A}
		\Choices{0.5$^\circ$C/100m}{0.56$^\circ$C/100m}{0.65$^\circ$C/100m}{1.0$^\circ$C/100m}
		
		\question
		按照中国科学院的气候区划,把我国分成9个气候带和\selectline 气候大区。\selectbracket{B}
		\Choices{9}{18}{28}{49}
		
		\question
		暖气团向冷气团方向移动的锋称为\selectline。\selectbracket{A}
		\Choices{暖锋}{冷锋}{准静止锋}{锢囚锋}
		
		\question
		气候最主要的特征是由热量和水分状况反映的,对于自然界植物的分布来说,其中\selectline 起着主导作用\selectbracket{A}
		\Choices{热量}{水分}{肥力}{太阳辐射}
		
		\question
		在气压一定时,露点温度的高低只与\selectline 有关。\selectbracket{B}
		\Choices{空气温度}{空气中水汽含量}{空气的饱和差}{空气的相对湿度}
		
		\question
		由于水的热容量、导热率均大,所以灌溉后的潮湿土壤,白天和夜间的温度变化是\selectline。\selectbracket{D}
		\Choices{白天升高慢,夜间降温快}{白天升高快,夜间降温慢}{白天和夜间升温降温都快}{白天和夜间升温降温都慢}	
		
		%判断说明题
		\MainQuestion{判断说明题}{先判断命题正误,然后说明正确和错误的理由,判断一分,说明一分,判断错误,全题不得分,每小题2分,共10分。}
		%用exam的\question命令和parts环境生成各个题目和其子题目,在题后用Answers环境留下正确答案。
		\question
		地球大气中的水汽含量一般来说是低纬多于高纬,下层多于上层,夏季多于冬季。\selectbracket{\true}
		\begin{Answers}[2]
			大气中的水汽主要来自于下垫面,低纬降水多,蒸发量大。夏季降水多,植被生长茂盛。
		\end{Answers}
		
		\question
		冷气块和暖气块上升同样的高度,冷气块的气压下降较少。\selectbracket{\flase}
		\begin{Answers}[2]
			冷气块密度大,气压下降多。
		\end{Answers}
		
		\question
		水汽凝结的条件是:$e\leqslant E$,有凝结核。\selectbracket{\flase}
		\begin{Answers}[2]
			因为$e\leqslant E$表示水汽饱和或不饱和,而产生凝结时应是水汽饱和或过饱和。
		\end{Answers}
		
		\question
		紧湿土壤,春季升温和秋季降温均比干松土壤要慢。\selectbracket{\true}
		\begin{Answers}[2]
			因为紧湿土壤的容积热量和导热率较干松土壤大。
		\end{Answers}
		
		\question
		南坡上日照时间随坡度变化,当坡度增加$1^\circ$其日照时间相当于原坡度北移纬度$1^\circ$的水平面上的日照时间。\selectbracket{\flase}
		\begin{Answers}[2]
			南坡上当当坡度增加$1^\circ$,其日照时间相当于原坡度南移纬度$1^\circ$的水平面上的日照时间。
		\end{Answers}
		%简答题
		\MainQuestion{简答题}{回答要点,并做简明扼要的解释。每小题5分,共40分}
		\question
		写出地面热量平衡方程并解释各项的含义。
		\begin{Answers}
			地面热量平衡方程:$R=P+LE+B$。R为净辐射,P为感热通量,LE为潜热通量,B为土壤热通量。
		\end{Answers}
		
		\question
		简要叙述大气气候的特征。
		\begin{Answers}[2]
			大陆性气候特征:气温日、年较差大、时相早;春温高于秋温,湿度小,降水量少而四季分配不均匀、日照多。
		\end{Answers}
		
		\question
		二氧化碳和臭氧有什么生态意义?
		\begin{Answers}
			$CO_2$的生态意义:1.光合作用的原料;2.影响地面和大气温度的变化。
			
			$O_2$的生态意义:1.吸收紫外线,保护地球生命;2.影响大气温度的垂直分布。
		\end{Answers}
		
		\question
	    湿绝热直减率为什么小于干绝热直减率?
	    \begin{Answers}
	    	1.饱和湿空气上升时,水汽凝结释放热量,温度降低较不饱和空气少;
	    	
	    	2.饱和湿空气下沉时,水汽蒸发消耗热量,温度升高值较不饱和空气少。
	    \end{Answers}
	    
		\question
		人类活动如何影响气候?
		\begin{Answers}
			1.改变了下垫面的性质;
			2.改变了大气成分;
			3.人工潜热的释放。
		\end{Answers}
		
		\question
		什么叫干热风?我国北方干热风的主要类型有哪些?
		\begin{Answers}
			干热风是指高温,低湿,并伴有一定风力的大气干旱现象。我国北方干热风的类型有:高温低湿型,雨后酷热型和旱风型。
		\end{Answers}
		
		\question
		按照三圈环流理论,北半球有几个气压带和几个风带?各是什么?
		\begin{Answers}
			按照三圈环流理论,北半球有赤道低压带、副热带高压带、副极地低压带、极地高压带四个气压带和信风带、盛行西风带、极地东风带三个风带。
		\end{Answers}
		
		\question
		日出日落是太阳为什么成红色?
		\begin{Answers}
			太阳初生和降落时,太阳辐射穿过大气的路程很长,在漫长的路程中太阳直接辐射的短波光几乎被散射殆尽,达到我们视野的只剩下红光。
		\end{Answers}
		
		%计算题
		\MainQuestion{计算题}{要求写出主要计算步骤和结果,每小题5分,共10分。}
		\question[5]
		已知相对湿度$r=40\%$,露点温度$td=2^\circ C$,求饱和差d。
		\begin{Answers}[8]
			\begin{gather*}
				R=E_0 10^{\frac{at}{b+c}}=6.11\times 10^{\frac{20a}{b+20}} \\
				E=\frac{ey}{40\%}=\frac{5}{2}e \\
				d=E-e=\frac{3}{2}e=\frac{3}{2}\times 6.11\times 10^{\frac{20a}{b+20}}=9.17\times 10^{\frac{20a}{b+20}}
			\end{gather*}
		\end{Answers}
		
		\question[5]
		某地地理纬度等于35N,5月1日的赤纬为15N,试计算该日的昼长为多少小时?
		\begin{Answers}[6]
			据$\cos\omega_0=-tg\varPhi tg\delta$,把$\varPhi=35^\circ$,$\delta=15^\circ$代入求得$\omega_0=100.8^\circ$。昼长$=\dfrac{2\omega_0}{15}=\dfrac{2\times100.8}{15}=13.44h$。
		\end{Answers}
	\end{questions}
\end{document}