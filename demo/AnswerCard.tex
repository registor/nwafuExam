% nwsuafexam文档类各个选项的含义:
% answers     是否显示答案(缺省为不显示)
% sealed      是否为密封试卷(缺省为普通)
% contitemcnt 是否为各小题题目进行连续编号(缺省为不连续编号)
% prescorebox 显示前置评分框(缺省为不显示)。
% cancelspace 忽略答题空白(在answers下此项无效)
% Tailscore   分值显示位置(题目开头或者题目右边界,缺省为题目开头)
\documentclass[sealed,contitemcnt]{nwsuafexam}%当调用或者取消一些选项参数时可能会造成试卷总页数改变,因此建议在调用或取消sealed、cancelspace、answers三个参数中的一个或几个时,先执行“工具”->“清理辅助文件”操作后再编译,否则会造成页数错乱。
\begin{document}
	%以test2.tex为例制作答题卡	
	% 设置试卷基本信息
	\papertitle{西北农林科技大学本科课程答题卡}
	\studyyear{2013}{2014}
	\semester{2}
	\subject{农林气象学}
	\testmethod{闭}
	\proteacher{张XX}
	\checkteacher{李XX}
	\papercategory{B}
	% 生成标题
	\maketitle
	%综合得分表
	{\heiti
		\begin{tabularx}{\textwidth}{|*{8}{>{\centering\arraybackslash}X|}}
			\hline
			题号 & 一 & 二 & 三 & 四 & 五 & 总分 & 审核人\\
			\hline
			得分 &    &    &    &    &    &     &      \\
			\hline
		\end{tabularx}}
	\raggedright %左对齐,误删,否则造成排版错误
	%填空题
	\MainQuestion{填空题}{每空2分,共20分}
	\tkfour{短波;长波}{自由大气}{小;小}{周期和范围}
	\tkfour{降温;100\%}{副热带}{$32.2^\circ$}{雨热同季}
	\tkone{气象要素}\tkone{绝对稳定}
	%单项选择题
	\MainQuestion{单项选择题}{每小题1分,共20分}
	\abcdfour{B}{C}{C}{B}
	\abcdfour{B}{D}{C}{C}
	\abcdfour{C}{D}{A}{B}
	\abcdfour{A}{A}{B}{D}
	%判断说明题
	\MainQuestion{判断说明题}{本大题共10题,每题2分,共20分。请将\true 或\flase 填入相应的括号内。填错或不填均不得分。}
	\tfreason{大气中的水汽主要来自于下垫面,低纬降水多。夏季降水多,植被生长茂盛。}{\true}
	\tfreason{冷气块密度大,气压下降多。}{\flase}
	\tfreason{因为$e\leqslant E$表示水汽饱和或不饱和,而产生凝结时应是水汽饱和或过饱和。}{\flase}
	\tfreason{因为紧湿土壤的容积热量和导热率较干松土壤大。}{\true}
	\tfreason{南坡上坡度增加$1^\circ$,其日照时间等于原坡度南移纬度$1^\circ$的水平面上日照时间。}{\flase}
	%简答题
	\MainQuestion{简答题}{请做简明扼要的解释。每小题5分,共40分}
	\Subitem[2]{地面热量平衡方程:$R=P+LE+B$。R为净辐射,P为感热通量,LE为潜热通量,B为土壤热通量。}%每两个\Subitem之间留一个空行,否则出现排版错误
		
	\Subitem[2]{大陆性气候特征:气温日、年较差大、时相早;春温高于秋温,湿度小,降水量少而四季分配不均匀、日照多。}
	
	\Subitem[2]{$CO_2$的生态意义:1.光合作用的原料;2.影响地面和大气温度的变化。
		
		$O_2$的生态意义:1.吸收紫外线,保护地球生命;2.影响大气温度的垂直分布。}
	
	\Subitem[2]{1.饱和湿空气上升时,水汽凝结释放热量,温度降低较不饱和空气少;
		
		2.饱和湿空气下沉时,水汽蒸发消耗热量,温度升高值较不饱和空气少。}
	
	\Subitem[2]{1.改变了下垫面的性质;2.改变了大气成分;3.人工潜热的释放。}
	
	\Subitem[2]{干热风是指高温,低湿,并伴有一定风力的大气干旱现象。我国北方干热风的类型有:高温低湿型,雨后酷热型和旱风型。}
	
	\Subitem[2]{按照三圈环流理论,北半球有赤道低压带、副热带高压带、副极地低压带、极地高压带四个气压带和信风带、盛行西风带、极地东风带三个风带。}
	
	\Subitem[2]{太阳初生和降落时,太阳辐射穿过大气的路程很长,在漫长的路程中太阳直接辐射的短波光几乎被散射殆尽,达到我们视野的只剩下红光。}
	%计算题
	\MainQuestion{计算题}{请写出计算步骤和结果,每小题5分,共10分。}
	\Subitem[7]{\begin{gather*}
		R=E_0 10^{\frac{at}{b+c}}=6.11\times 10^{\frac{20a}{b+20}} \\
		E=\frac{ey}{40\%}=\frac{5}{2}e \\
		d=E-e=\frac{3}{2}e=\frac{3}{2}\times 6.11\times 10^{\frac{20a}{b+20}}=9.17\times 10^{\frac{20a}{b+20}}
		\end{gather*}}
	
	\Subitem[7]{据$\cos\omega_0=-tg\varPhi tg\delta$,把$\varPhi=35^\circ$,$\delta=15^\circ$代入求得$\omega_0=100.8^\circ$。\\
		昼长$=\frac{2\omega_0}{15}=\frac{2\times100.8}{15}=13.44h$。}
\end{document}
