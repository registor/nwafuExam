% nwsuafexam文档类各个选项的含义:
% answers     是否显示答案(缺省为不显示)
% sealed      是否为密封试卷(缺省为普通)
% contitemcnt 是否为各小题题目进行连续编号(缺省为不连续编号)
% prescorebox 显示前置评分框(缺省为不显示)。
% cancelspace 忽略答题空白(在answers下此项无效)
% Tailscore   分值显示位置(题目开头或者题目右边界,缺省为题目开头)
\documentclass[prescorebox,contitemcnt,answers]{nwsuafexam}%当调用或者取消一些选项参数时可能会造成试卷总页数改变,因此建议在调用或取消sealed、cancelspace、answers三个参数中的一个或几个时,先执行“工具”->“清理辅助文件”操作后再编译,否则会造成页数错乱。
\begin{document}	
	% 设置试卷基本信息
	\papertitle{西北农林科技大学本科课程考试试卷}
	\studyyear{2010}{2011}
	\semester{2}
	\subject{材料力学}
	\testmethod{闭}
	\proteacher{张XX}
	\checkteacher{李XX}
	\papercategory{A}
	% 生成标题
	\maketitle
	%注意事项(提供两个环境参数,分别为总分数和考试时间,环境中默认已经定义了“本试卷共XX道试题, 满分XX分,考试时间XX分钟.”格式的一条注意事项,其余事项可在环境中使用\item命令输入)
	\begin{notice}{100}{120}
		\item 学生在答题前请先填写专业、学号、学院、姓名等基本信息。
	\end{notice}
	
	%综合得分表
	{\heiti
		\begin{tabularx}{\textwidth}{|*{7}{>{\centering\arraybackslash}X|}}
			\hline
			题号 & 一 & 二 & 三 & 四 & 总分 & 审核人\\
			\hline
			得分 &    &    &    &    &     &      \\
			\hline
		\end{tabularx}}
	\begin{questions}
		%填空题
		\MainQuestion{填空题}{本题共有5个小题,每空1分,共10分}
		\question
	    杆件变形的四种基本形式有:\blank{压缩}、\blank{剪切}、\blank{扭转}、\blank{弯曲}。
		
		\question
		标距为100mm的标准试样,直径为10mm,拉断后测得伸长后的标距为123mm,缩颈处的最小直径为7mm,则该材料的伸长率$\delta=$\blank{$23\%$},断面收缩率$\psi=$\blank{$51\%$}。
		
		\question
		从强度角度出发,截面积相同的矩形杆件和圆形杆件,\blank{矩形杆件} 更适合做承受弯曲变形为主的梁。
		
		\question
		某点的应力状态如图示,则主应力为:$\zeta_1=$\blank{$80MPa$};$\zeta_2=$\blank{$30MPa$}。
		\onegraphc{tu1}
		
		\question
		平面图形对过其形心轴的静矩\blank{=}0(请填入 =,>, <)
		
		%单项选择题
		\MainQuestion{单项选择题}{每小题2分,共20分}
		\question
		受扭圆轴,上面作用的扭矩T不变,当直径减小一半时,该截面上的最大切应力与原来的最大切应力之比为\selectline。\selectbracket{D}
		\Choices{2}{4}{6}{8}
		
		\question
		图示为一端固定的橡胶板条,若在加力前在板表面划条斜直线AB,那么加轴向拉力后AB线所在位置是\selectline ?(其中ab//AB//ce)\selectbracket{B}
		\onegraphc[0.3]{tu2}
		\Choices{ab}{ae}{ce}{ed}
		
		\question
		根据切应力互等定理,图示的各单元体上的切应力正确的是\selectline。\selectbracket{A}
		\onegraphchoices{tu3}
		
		\question
		在平面图形的几何性质中,\selectline 的值可正、可负、也可为零。 \selectbracket{D}
		\Choices{静矩和惯性矩}{极惯性矩和惯性矩}{惯性矩和惯性积}{静矩和惯性积}
		
		\question
		受力情况相同的三种等截面梁,用$(\zeta_{max})_1$、$(\zeta_{max})_2$、$(\zeta_{max})_3$分别表示三根梁内横截面上的最大正应力,则下列说法正确的是\selectline。 \selectbracket{C}
		\onegraphc[0.5]{tu4}
		\Choices{$(\zeta_{max})_1=(\zeta_{max})_2=(\zeta_{max})_3$}{$(\zeta_{max})_1<(\zeta_{max})_2=(\zeta_{max})_3$}{$(\zeta_{max})_1=(\zeta_{max})_2<(\zeta_{max})_3$}{$(\zeta_{max})_1<(\zeta_{max})_2<(\zeta_{max})_3$}
		
		\question
		在图示矩形截面上,剪力为Fs,欲求m-m线上的切应力,则公式$\tau=\dfrac{F_s\bullet S^*_z}{BI_z}$中, 下列说法 正确的是\selectline \selectbracket{D}
		\onegraphc[0.2]{tu5}
		\Choices{$S^*_z$为截面的阴影部分对$Z'$轴的静矩,$B=\delta$。}{$S^*_z$为截面的整个部分对$Z'$轴的静矩,$B=\delta$。}{$S^*_z$为截面的整个部分对$Z$轴的静矩,$B=\delta$。}{$S^*_z$为截面的阴影部分对$Z$轴的静矩,$B=\delta$。}
		
		\question
		已知梁的$EI_z$为常数,长度为l ,欲使两的挠曲线在$x=l/3$处出现一拐点,则比值$m_1/m_2=$\selectline \selectbracket{C}
		\onegraphc{tu6}
		\Choices{2}{3}{1/2}{1/3}
		
		\question
		如图所示单向均匀拉伸的板条。若受力前在其表面画上两个正方形a和b,则受力后正方形a、b分别变为\selectline。 不会产生温度应力。 \selectbracket{D}
		\Choices{正方形、正方形}{正方形、菱形}{矩形、菱形}{矩形、正方形}
		\onegraphc{tu7}
		
		\question
		低碳钢试样拉伸至屈服时,有以下结论,请判断哪个是正确的\selectline。 \selectbracket{C}
		\Choices{应力和塑性变形很快增加,因而认为材料失效;}{应力和塑性变形虽然很快增加,但不意味着材料失效;}{应力不增加,塑性变形很快增加,因而认为材料失效;}{应力不增加,塑性变形很快增加,但不意味着材料失效。}
		
		\question
		图示拉杆头和拉杆的横截面均为圆形,拉杆头剪切面积$A=$\selectline。\selectbracket{B}
		\onegraphc{tu8}
		\Choices{$\pi Dh$}{$\pi dh$}{$\pi d^2/4$}{$\pi(D^2-d^2)/4$}
		
		%分析作图题
		\MainQuestion{分析作图题}{每小题10分,共20分}
		\question[10]
		求做图示构件的内力图。
		\onegraphr{tu9}
		%当参考答案中需要插入图片时请用\answerwithgraph命令输入答案
		\answerwithgraph{
			剪力图 :\hfill (5分)
			\onegraphc{tu15}
			弯矩图:\hfill (5分)
			\onegraphc{tu16}
		}
		
		\question[10]
		图示矩形等截面梁,试比较水平放置与竖立放置时最大弯曲正应力的比值$\zeta_{\text{平}}/\zeta_{\text{立}}$,说明那种放置方式合理。
		\onegraphr{tu10}
		\begin{Answers}[5]
			由弯曲正应力$s=\dfrac{M}{W_z}$和矩形梁$W_z=\dfrac{bh^2}{6}$\hfill(4分)\\
			可知:$\dfrac{s_{\text{平}}}{s_{\text{立}}}=\dfrac{W_{z\text{立}}}{W_{z\text{平}}}=\dfrac{b\cdot(4b)^2}{4b\cdot b^2}=4$\hfill(4分)\\
			因此可知,梁竖立放置合理。\hfill(2分)
		\end{Answers}
		
		%计算题
		\MainQuestion{计算题}{共50分}
		\question[15]
		图示阶梯状直杆,若横截面积$A_1=200mm^2$,$A_2=300mm^2$,$A_3=400mm^2$。
		\begin{parts}
			\part[5]
		    试求横截面$1-1$, $2-2$, $3-3$上的轴力,并作轴力图;
			\part[5]
			求横截面$3-3$上的应力。
		\end{parts}
		\onegraphr{tu11}
	    \answerwithgraph{
	    	$F_{N1}=-20KN(\text{压})$\\
	    	$F_{N2}=-10KN(\text{压})$\\
	    	$F_{N3}=10KN(\text{压})$\hfill (5分)\\
	    	轴力图:\hfill (5分)
	    	\onegraphc{tu17}
	    	$s_3=\dfrac{F_{N3}}{A_3}=\dfrac{10'10^3}{40'10^{-6}}=25MPa$\hfill (5分)
	    }
		
		\question[20]
		已知某受力构件上危险点应力状态如图所示,已知材料的弹性模量$E=200GPa$,泊松比$\mu=0.3$,求该单元体的主应力、最大主应变及最大切应力(应力单位为MPa)。
		\onegraphc[0.3]{tu14}
		\answerwithgraph{
			由题知$\zeta=50MP$是主应力之一,考虑其它两对平面,可视为平面应力,则应力圆为:
			\onegraphc[0.3]{tu18}\hfill(5分)\\
			解得其它两个主应力为80MPa和-20MPa,因此三个主应力分别为:\\
			$\zeta_1=80MPa$,$\zeta_2=50MPa$,$\zeta_3=-20MPa$\hfill(5分)\\
			最大切应力为$\eta=(\zeta_1-\zeta_3)/2=50MPa$\hfill(5分)\\
			有广义胡克定律知最大主应变为:$\varepsilon_1=[\zeta_1-\mu(\zeta_2+\zeta_3)]/E=0.355\times 10^{-3}$\hfill(5分)\\
		}
		
		\question[15]
		图示实心轴和空心轴通过牙嵌式离合器连接在一起。已知轴的转速$n=100r/min$,传递的功率$P=7.5kw$,材料的许用应力$[\eta]=40MPa$,空心圆轴的内外径之比$d_2=0.5D_2$。试选择实心轴的直径$d_1$和空心轴内外径$D_2$。
		\onegraphr{tu12}
		\begin{Answers}[5]
			轴所传递的扭矩为\\
			$T=9549\frac{P}{M}=9549\frac{7.5}{100}Nm=716Nm$\hfill (3分)\\
			由实心轴的强度条件 \\
			$t_{max}=\frac{T}{W_t}=\frac{16T}{pd^3_1}~[t]$\hfill (3分)\\
			可得实心圆轴的直径为 \\
			$d_1=\sqrt[3]{\frac{16T}{p[t]}}=\sqrt[3]{\frac{16'716}{p\cdot 40\cdot 10^6}}=45mm$\hfill (3分)\\
			由空心轴的强度条件$t_{max}=\dfrac{T}{W_t}=\dfrac{16T}{pD^3_2(1-0.5^4)}~[t]$\hfill (3分)\\
			空心圆轴的外径为 \\
			$D_2=\sqrt[3]{\frac{16T}{p[t](1-0.5^4)}}-\sqrt[3]{\frac{16'716}{p\cdot 40\cdot 10^6(1-0.5^4)}}=46mm$\hfill (3分)\\
		\end{Answers}
		\end{questions}
\end{document}