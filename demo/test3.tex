% nwsuafexam文档类各个选项的含义:
% answers     是否显示答案(缺省为不显示)
% sealed      是否为密封试卷(缺省为普通)
% contitemcnt 是否为各小题题目进行连续编号(缺省为不连续编号)
% prescorebox 显示前置评分框(缺省为不显示)。
% cancelspace 忽略答题空白(在answers下此项无效)
% Tailscore   分值显示位置(题目开头或者题目右边界,缺省为题目开头)
\documentclass[sealed,answers,prescorebox]{nwsuafexam}%当调用或者取消一些选项参数时可能会造成试卷总页数改变,因此建议在调用或取消sealed、cancelspace、answers三个参数中的一个或几个时,先执行“工具”->“清理辅助文件”操作后再编译,否则会造成页数错乱。
\begin{document}	
	% 设置试卷基本信息
	\papertitle{西北农林科技大学本科课程考试试卷}
	\studyyear{2011}{2012}
	\semester{1}
	\subject{C++面向程序设计}
	\testmethod{闭}
	\proteacher{张XX}
	\checkteacher{李XX}
	\papercategory{A}
	% 生成标题
	\maketitle
	%注意事项(提供两个环境参数,分别为总分数和考试时间,环境中默认已经定义了“本试卷共XX道试题, 满分XX分,考试时间XX分钟.”格式的一条注意事项,其余事项可在环境中使用\item命令输入)
	\begin{notice}{100}{120}
		\item 学生在答题前请先填写专业、学号、学院、姓名等基本信息。
	\end{notice}
	
	%综合得分表
	{\heiti
		\begin{tabularx}{\textwidth}{|*{8}{>{\centering\arraybackslash}X|}}
			\hline
			题号 & 一 & 二 & 三 & 四 & 五 & 总分 & 审核人\\
			\hline
			得分 &    &    &    &    &    &     &      \\
			\hline
		\end{tabularx}}
	%输入试卷题目
	\begin{questions}
		%单项选择题
		\MainQuestion{单项选择题}{本大题共10小题,每小题2分,共20分。在每小题列出的四个备选项中,只有一个是符合题目要求的,请将其代码填写在题后的括号内。错选、多选或未选均无分。}
		\question
		说明虚函数的关键字是\selectline。\selectbracket{B}
		\Choices{inline}{virtual}{define}{static}
		
		\question
		在每个C++程序中都必须包含有这样一个函数,该函数的函数名为\selectbracket{A}
		\Choices{main}{MAIN}{name}{function}
		
		\question
		cout是某个类的标准对象的引用,该类是\selectline。\selectbracket{A}
		\Choices{ostream}{istream}{stdout}{stdin}
		
		\question
		如果在类外有函数调用CPoint::func();则函数func()是类CPoint的\selectbracket{C}
		\Choices{私有静态成员函数}{公有非静态成员函数}{公有静态成员函数}{友元函数}
		
		\question
		如果class类中的所有成员在定义时都没有使用关键字public、private或protected,则所有成员缺省定义为\selectline。\selectbracket{C}
		\Choices{public}{protected}{private}{static}
		
		\question
		一个类的所有对象共享的是\selectline。\selectbracket{D}
		\Choices{私有数据成员}{公有数据成员}{保护数据成员}{静态数据成员}
		
		\question
		动态联编所支持的多态性称为\selectline。\selectbracket{D}
		\Choices{虚函数}{继承}{编译时多态性}{运行时多态性}
		
		\question
		定义类模板时要使用关键字\selectline。\selectbracket{D}
		\Choices{const}{new}{delete}{template}
		
		\question
		对虚基类的定义\selectline。\selectbracket{A}
		\Choices{不需要使用虚函数}{必须使用虚函数}{必须使用private}{必须使用public}
		
		\question
		类型转换函数\selectline。\selectbracket{A}
		\Choices{不能带有参数}{只能带一个参数}{只能带2个参数}{只能带3个参数}
		
		%判断正误题
		\MainQuestion{判断正误题}{本大题共10小题,每小题2分,共20分。判断正误,在题后的括号内,正确的划上“\true”错误的划上“\flase”。}
		\question
		虚函数在基类和派生类之间定义,且要求函数原型完全一致。\selectbracket{\true}
		
		\question
		抽象类可以用来直接创建对象。\selectbracket{\flase}
		
		\question
		内联函数中可以出现递归语句。\selectbracket{\flase}
		
		\question
		模板类与类模板的意义完全相同。\selectbracket{\flase}
		
		\question
		常对象只能调用常成员函数。\selectbracket{\true}
		
		\question
		重载函数要求函数有相同的函数名,但具有不同的参数序列。\selectbracket{\true}
		
		\question
		不可以定义抽象类的对象。\selectbracket{\true}
		
		\question
		内联函数的定义必须出现在第一次调用内联函数之前。\selectbracket{\true}
		
		\question
		模板函数与函数模板的意义完全相同。\selectbracket{\flase}
		
		\question
    	只有常成员函数才可以操作常对象。\selectbracket{\true}
		
		%填空题
	    \MainQuestion{填空题}{本大题共10小题,每小题2分,共20分。不写解答过程,将正确的答案写在每小题的空格内。错填或不填均无分。}
	    \question
	    在用C++进行程序设计时,最好用\blank{new} 代替malloc。
	    
	    \question
	    函数模板紧随template之后尖括号内的类型参数要寇以保留字\blank{class}。
	    
	    \question
	    编译时多态性可以用\blank{重载} 函数实现。
	     
	    \question
	    拷贝构造函数用它所在类的\blank{对象} 作为参数。
	    
	    \question
	    用关键字static修饰的类的成员称为\blank{静态} 成员。
	    
	    \question
	    重载运算符“+”的函数名为\blank{operator+}。
	    
	    \question
	    设函数max是由函数模板实现的,并且max(3.5, 5)和max(3, 5)都是正确的函数调用,则此函数模板具有\blank{2} 个类型参数。
	    
	    \question
	    在C++中,函数重载与虚函数帮助实现了类的\blank{多态} 性。
	    
	    \question
	    由static修饰的数据成员为该类的所有对象\blank{共享}。
	    
	    \question
	    重载函数在参数类型或参数个数上不同,但\blank{函数名} 相同。
	    
	    %程序分析题
	    \MainQuestion{程序分析题}{本大题共4小题,每小题5分,共20分。给出下面各程序的输出结果。}
	    \question
	    阅读下面程序,写出输出结果。
	    \begin{flushleft}
	    	\wuhao
	    	 \begin{verbatim}
	    	 #include <iostream>
	    	 using namespace std;
	    	 class CArray
	    	 {
	    	 public:
	    	 CArray(int iArray[], int iSize):m_pArray(iArray), m_iSize(iSize)
	    	 {
	    	 }
	    	 int GetSize()
	    	 {
	    	 return m_iSize;
	    	 }
	    	 int &operator[](int iIndex)
	    	 {
	    	 return m_pArray[iIndex - 1];
	    	 }
	    	 private:
	    	 int *m_pArray;			// 指向一个数组空间
	    	 int m_iSize;				// 数组元素个数
	    	 };
	    	 int main()
	    	 {
	    	 int s[]={3, 7, 2, 1, 5};
	    	 CArray oArray(s, 5);
	    	 oArray[1] = 9;
	    	 for (int i = 1; i <= 5; i++)
	    	 {
	    	 cout << oArray[i] << "  ";
	    	 }
	    	 cout << endl;
	    	 return 0;
	    	 }
	    	 \end{verbatim}
	    \end{flushleft}
	    上面程序的输出结果为:
	    \begin{Answers}[1]
	    	9~7~2~1~5
	    \end{Answers}
	    
	    \question
	    阅读下面程序,写出输出结果。
	    \begin{flushleft}
	    	\wuhao
		    \begin{verbatim}
	    	#include <iostream>
	    	using namespace std;
	    	
	    	template <class Type>
	    	void Print(Type a[], int n)
	    	{
	    	for (int i = 0; i < n; i++)
	    	{
	    	cout << a[i] << "  ";
	    	}
	    	}
	    	int main()
	    	{
	    	int a[] = { 5, 6, 8};
	    	double b[] = {6.8, 9.6};
	    	
	    	Print(a, sizeof(a) / sizeof(int));
	    	Print(b, 2);
	    	cout << endl;
	    	
	    	return 0;
	    	}
		    \end{verbatim}
		\end{flushleft}
	    上面程序的输出结果为:
	    \begin{Answers}[1]
	    	5~6~8~6.8~9.6
	    \end{Answers}
	    
	    \question
	    阅读下面程序,写出输出结果。
	    \begin{flushleft}
	    	\wuhao
		    \begin{verbatim}
	    	#include <iostream>
	    	using namespace std;
	    	class CTest
	    	{
	    	public:
	    	CTest(int iVar):m_iVar(iVar)
	    	{
	    	m_iCount++;
	    	}
	    	~CTest()
	    	{
	    	}
	    	void Print() const;
	    	static int GetCount()
	    	{
	    	return m_iCount;
	    	}
	    	private:
	    	int m_iVar;
	    	static int m_iCount;
	    	};
	    	int CTest::m_iCount = 0;
	    	void CTest::Print() const
	    	{
	    	cout << this->m_iVar << "  " << this->m_iCount << "  ";
	    	}
	    	int main()
	    	{
	    	CTest oTest1(6);
	    	oTest1.Print();
	    	CTest oTest2(8);
	    	oTest2.Print();
	    	cout << CTest::GetCount();
	    	cout << endl;
	    	return 0;
	    	}
		    \end{verbatim}
		\end{flushleft}
	    上面程序的输出结果为:
	    \begin{Answers}[1]
	    	6~1~8~2~2
	    \end{Answers}
	    
	    \question
	    阅读下面程序,写出输出结果。
	    \begin{flushleft}
	    	\wuhao
		    \begin{verbatim}
	    	#include <iostream>
	    	using namespace std;
	    	class CTest
	    	{
	    	public:
	    	CTest(int iX = 0, int iY = 0, int iZ = 0):m_iZ(iZ)
	    	{
	    	m_iX = iX;
	    	m_iY = iY;
	    	}
	    	void Print()
	    	{
	    	cout << m_iX << endl;
	    	cout << m_iY << endl;
	    	}
	    	void Print() const
	    	{
	    	cout << m_iZ << endl;
	    	}
	    	private:
	    	int m_iX, m_iY;
	    	const int m_iZ;
	    	};
	    	int main()
	    	{
	    	CTest oTest1;
	    	oTest1.Print();
	    	CTest oTest2(1, 6, 8);
	    	oTest2.Print();
	    	const CTest oTest3(6, 0, 18);
	    	oTest3.Print();
	    	cout << endl;
	    	return 0;
	    	}
		    \end{verbatim}
		\end{flushleft}
	    上面程序的输出结果为:
	    \begin{Answers}[1]
	    	0~0~1~6~18
	    \end{Answers}
	    
	    %编程题
	    \MainQuestion{编程题}{本大题共2个小题,每小题10分,共20分}
	    \question
	    编写一个函数模板,用于求参数的绝对值,并编写测试程序进行测试。
	    函数模板声明如下:
	    \begin{flushleft}
	    	\wuhao
	    	\begin{verbatim}
	    	template <class Type>
	    	Type Abs(Type tVar)
	    	\end{verbatim}
	    \end{flushleft}
	    \begin{Answers}[0]
	    	\begin{flushleft}
	    		\wuhao
	    		\begin{verbatim}
	    		#include <iostream>
	    		using namespace std;
	    		template <class Type>
	    		Type Abs(Type tVar)
	    		{
	    		if (tVar >= 0) return tVar;
	    		else return -tVar;
	    		}
	    		int main()
	    		{cout << Abs(5) << endl;
	    		cout << Abs(-5) << endl;
	    		cout << Abs(2.5) << endl;
	    		cout << Abs(-2.5) << endl;
	    		return 0;}
	    		\end{verbatim}
	    	\end{flushleft}
	    \end{Answers}
	    
	    \question
	    定义一个复数类CComplex,定义带有2个参数(其中一个为缺省参数)的构造函数,显示复数值的函数Show(), 重载“+”运算符(用成员函数实现),并编写测试程序进行测试。
	    \begin{Answers}[0]
	    	\begin{flushleft}
	    		\wuhao
	    		\begin{verbatim}
	    		#include <iostream>
	    		using namespace std;
	    		class CComplex
	    		{
	    		public:
	    		CComplex(double dReal, double dImage = 0)
	    		{
	    		m_dReal = dReal;
	    		m_dImage = dImage;
	    		}
	    		void Show()
	    		{
	    		cout << m_dReal;
	    		if (m_dImage > 0)
	    		{cout << "+" << m_dImage << "i" << endl;}
	    		else if(m_dImage < 0)
	    		{cout << "-" << -m_dImage << "i" << endl;}
	    		}
	    		CComplex operator+(const CComplex &obj)
	    		{
	    		CComplex objTemp(m_dReal + obj.m_dReal, m_dImage + obj.m_dImage);
	    		return objTemp;
	    		}
	    		private:
	    		double m_dReal, m_dImage;
	    		};
	    		int main()
	    		{
	    		CComplex obj1(2, 6), obj2(3, 8), obj3(0);
	    		obj1.Show();
	    		obj2.Show();
	    		obj3.Show();
	    		obj3 = obj1 + obj2;
	    		obj3.Show();
	    		return 0;
	    		}
	    		\end{verbatim}
	    	\end{flushleft}	
	    \end{Answers}
	\end{questions}
\end{document}