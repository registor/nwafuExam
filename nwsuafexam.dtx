% \iffalse meta-comment
%
% Copyright (C) 2015 by Geng Nan <nangeng@qq.com>
%
% This file is part of the nwsuafexam package project.
% ---------------------------------------------------
%
% It may be distributed under the conditions of the LaTeX Project Public
% License, either version 1.2 of this license or (at your option) any
% later version. The latest version of this license is in
%    http://www.latex-project.org/lppl.txt
% and version 1.2 or later is part of all distributions of LaTeX
% version 1999/12/01 or later.
%
%<*!(cfg|fd)>
% \fi
%
%% \CharacterTable
%%  {Upper-case    \A\B\C\D\E\F\G\H\I\J\K\L\M\N\O\P\Q\R\S\T\U\V\W\X\Y\Z
%%   Lower-case    \a\b\c\d\e\f\g\h\i\j\k\l\m\n\o\p\q\r\s\t\u\v\w\x\y\z
%%   Digits        \0\1\2\3\4\5\6\7\8\9
%%   Exclamation   \!     Double quote  \"     Hash (number) \#
%%   Dollar        \$     Percent       \%     Ampersand     \&
%%   Acute accent  \'     Left paren    \(     Right paren   \)
%%   Asterisk      \*     Plus          \+     Comma         \,
%%   Minus         \-     Point         \.     Solidus       \/
%%   Colon         \:     Semicolon     \;     Less than     \<
%%   Equals        \=     Greater than  \>     Question mark \?
%%   Commercial at \@     Left bracket  \[     Backslash     \\
%%   Right bracket \]     Circumflex    \^     Underscore    \_
%%   Grave accent  \`     Left brace    \{     Vertical bar  \|
%%   Right brace   \}     Tilde         \~}
%%
%
% \CheckSum{0}
%
% \iffalse meta-comment
%</!(cfg|fd)>
%
%<*driver>
\ProvidesFile{nwsuafexam.dtx}
%</driver>
%
%<cls>\NeedsTeXFormat{LaTeX2e}[2007/10/19]
%<cls>\ProvidesClass{nwsuafexam}
%<cfg>\ProvidesFile{nwsuafexam.cfg}
  [2016/01/04 v0.1 nwsuafexam
%<cls>   document class]
%<cfg>   configuration file]
%
%<*driver>
   bundle source file]
%</driver>
%
%<*driver>
\documentclass[a4paper]{ltxdoc}
\usepackage{ctex}
\usepackage{hyperref}
\usepackage{amsmath,amssymb}
 \topmargin 0.5 true cm
 \oddsidemargin 1 true cm
 \evensidemargin 1 true cm
 \textheight 21 true cm
 \textwidth 14 true cm
\EnableCrossrefs
 %\DisableCrossrefs % Say \DisableCrossrefs if index is ready
\CodelineIndex
\RecordChanges      % Gather update information
 %\OnlyDescription  % comment out for implementation details
 %\OldMakeindex     % use if your MakeIndex is pre-v2.9
\hypersetup{colorlinks,linkcolor=blue,citecolor=blue}
\begin{document}  
 \DocInput{nwsuafexam.dtx}
\end{document}
%</driver>
%
% \fi
%
%
% \changes{v0.1}{2016/01/04}{Initial version}
%
%
% \DoNotIndex{\begin,\end,\begingroup,\endgroup}
% \DoNotIndex{\ifx,\ifdim,\ifnum,\ifcase,\else,\or,\fi}
% \DoNotIndex{\let,\def,\xdef,\newcommand,\renewcommand}
% \DoNotIndex{\expandafter,\csname,\endcsname,\relax,\protect}
% \DoNotIndex{\Huge,\huge,\LARGE,\Large,\large,\normalsize}
% \DoNotIndex{\small,\footnotesize,\scriptsize,\tiny}
% \DoNotIndex{\normalfont,\bfseries,\slshape,\interlinepenalty}
% \DoNotIndex{\hfil,\par,\vskip,\vspace,\quad}
% \DoNotIndex{\centering,\raggedright}
% \DoNotIndex{\c@secnumdepth,\@startsection,\@setfontsize}
% \DoNotIndex{\ ,\@plus,\@minus,\p@,\z@,\@m,\@M,\@ne,\m@ne}
% \DoNotIndex{\@@par}
%
%
% \GetFileInfo{nwsuafexam.dtx}
%
%
% \MakeShortVerb{\|}
% \setcounter{StandardModuleDepth}{1}
%
%
% \newcommand{\ctex}{\texttt{ctex}}
% \newcommand{\nwsuafexam}{\texttt{nwsuafexam}}
% \newcommand{\exam}{\texttt{exam}}
% \newcommand{\colin}{\texttt{colinexam}}
% \newcommand{\bhc}{\texttt{BHCexam}}
% \newcommand{\nwsuaf}{{\heiti 西北农林科技大学}}
% \newcommand{\cie}{{\heiti 信息工程学院}}
% \newcommand{\nwcie}{{\heiti 西北农林科技大学信息工程学院}}
%
%
% \setlength{\parskip}{0.75ex plus .2ex minus .5ex}
% \renewcommand{\baselinestretch}{1.2}
%
% \makeatletter
% \def\parg#1{\mbox{$\langle${\it #1\/}$\rangle$}}
% \def\@smarg#1{{\tt\string{}\parg{#1}{\tt\string}}}
% \def\@marg#1{{\tt\string{}{\rm #1}{\tt\string}}}
% \def\marg{\@ifstar\@smarg\@marg}
% \def\@soarg#1{{\tt[}\parg{#1}{\tt]}}
% \def\@oarg#1{{\tt[}{\rm #1}{\tt]}}
% \def\oarg{\@ifstar\@soarg\@oarg}
% \makeatother
%
%
% \title{\bf \nwsuafexam 宏包说明\thanks
%   {这是\nwcie 耿楠发布的\LaTeX 宏包。本文件版本号为\fileversion{},最后修改日期\filedate{}。}}
% \author{\it 耿楠\thanks{nangeng@qq.com}}
% \date{\small 日期:\today}
% \maketitle
%
% \begin{abstract}
% \nwsuafexam 宏包提供了一个\nwsuaf 试卷排版的\LaTeX{}文档类。
% 该宏包的主要目的是为了实现\nwsuaf 试卷排版中内容与格式
% 的分离,以期减轻\nwsuaf 广大教师的试卷排版工作。通过内容与格
% 式的分离,使广大教师能够更加专注于试卷内容的组织和编写。
%
% \nwsuafexam 包括\texttt{nwsuafexam.cls}文档类和配置文件
% \texttt{nwsuafexam.cfg}两个文件,\texttt{nwsuafexam.cls}文档类用
% 于进行试卷格式控制,\texttt{nwsuafexam.cfg}文件用于进行必须的变量赋
% 值设置。
%
% \nwsuafexam 宏包由\nwcie 耿楠制作并负责维护。
% \end{abstract}
%
% \tableofcontents
% \newpage
%
% \section{简介}
%
% 本宏包以 \exam 为底层文档类,通过修改由鲍宏昌开发的\bhc 宏包实现。
% \bhc 宏包的部分源代码来自于盖鹤麟开发的\colin。但不知道什么原因
% 盖鹤麟自2004年就一直没有发布更新,\colin 仍然使用CCT实现中文支持,而
% 缺乏对XeTeX的支持。2011年7月,鲍宏昌在 \colin 的基础上改用 \ctex 实
% 现中文支持,采用UTF8编码支持 XeLaTeX,并使用\texttt{doc}
% 和\texttt{docstrip}工具编写了宏包文档,增加了一些新
% 的功能,在此基础上,\nwcie 耿楠进一步对 \bhc 宏包进行了调整和修改,
% 使其符合\nwsuaf 教务处规定的试卷排版格式,并将宏包命名为\nwsuafexam。
%
% 本宏包延续了 \bhc 、 \colin 和\exam 的特点,能让一个刚刚接触
% \LaTeX 的初学者轻松地实现~\nwsuaf~试卷排版。希望\nwsuafexam 能够提高
% \nwsuaf 广大教师的工作效率,并把注意力放在试卷的内容上。
% 
% \nwsuafexam 由两个文件构成:文档类\texttt{nwsuafexam.cls}和配置文件
% \texttt{nwsuafexam.cfg},后者定义了一些常用的参数。
%
% {\kaishu
% 这两个文件可以通过用XeLaTeX编译\texttt{nwsuafexam.ins}文件来得到,
% 而这份说明文档可以通过用XeLaTeX编译\texttt{nwsuafexam.dtx}文件来得到。
% 编译说明文档需要\ctex{}宏包,为了生成正确的索引和版本记录,
% 需要使用如下命令}
% \begin{verbatim}
% makeindex -s gind.ist -o nwsuafexam.ind nwsuafexam.idx
% makeindex -s gglo.ist -o nwsuafexam.gls nwsuafexam.glo
% \end{verbatim}
%
% \section{一个简单的例子}
%
% 用\nwsuafexam 排版一张试卷其实很简单。如果你准备对试卷的排版进行
% 更细致的设置,那么请参考\exam 的文档。
%
% \subsection{\texttt{documentclass} 命令}
% \label{sec:BasicDocumentclass}
%
% 要使用\nwsuafexam 文档类,你的\verb"\documentclass" 命令应该是
% \begin{verbatim}
%  \documentclass{nwsuafexam}
% \end{verbatim}
% 同时,\nwsuafexam 文档类引入不同的参数,以实现不同的设置,如,当需要
% 使用密封试卷时,可以使用选项\verb"sealedpaper"
% \begin{verbatim}
% \documentclass[sealedpaper]{nwsuafexam}
% \end{verbatim}
% 更多的选项,请参考\ref{sec:Options}。
%
% \subsection{输出标题和考试须知}
% \DescribeMacro{\maketitle}
% \DescribeMacro{\notice}
% 在试卷上输出标题和考试须知
% \begin{verbatim}
%   \maketitle
%   \notice 
% \end{verbatim}
% 关于标题和考试须知中变量的设置,请参考\ref{sec:Variable}。
%
% \subsection{题目}
% \label{sec:Example}
% \DescribeMacro{\question} \DescribeMacro{\shortblk}
% \DescribeMacro{\singlelinech}
% \DescribeMacro{\part}在\verb"questions"环境中用\verb"\question"输入题
% 目。在\verb"parts"环境中用\verb"\part"输入大题的小问。
% 用\verb"\shortblk"、\verb"\middleblk"和\verb"\longblk"输入填空题的答
% 案,并确定下画线的长度。
% 用\verb"\singlelinech"、\verb"\doublelinesch"和\verb"\fourlinesch"输
% 入选择题的选项,并确定选项的显示方式。在\verb"questions"环境中
% 用\verb"\setmainitem"设置各题型的提示语,\verb"\mainitem"显示各题型的
% 提示语。
% \begin{verbatim}
%       \begin{questions}%               
%               \setmainitem{名词解释}{每小题2分,共30分,请为列出的名词进行解释}
%               \mainitem
%               \question 名词解释1 数组
%               \question 名词解释2 指针
%               
%               \setmainitem{填空题}{每空2分,共30分,请在每题空白处填写正确答案}
%               \mainitem
%               \question 这是第1 道填空题\shortblk{答案,不显示答案时显示短横线}
%               \question 这是第2 道填空题\middleblk{答案,不显示答案时显示横线}
%               \question 这是第3 道填空题\longblk{答案,不显示答案时显示长横线}
%               
%               \setmainitem{选择题}{每小题2分,共30分,请在每题后的括号里填写正确的选项}
%               \mainitem
%               \question 这是第1 道选择题, 四个选项显示在一行 \selectedbox{A} \\
%               \singlelinech{选项1}{选项2}{选项3}{选项4}
%               \question 这是第2 道选择题, 四个选项显示在两行 \selectedbox{A} \\
%               \doublelinech{选项1}{选项2}{选项3}{选项4}
%               \question 这是第3 道选择题, 四个选项显示在四行 \selectedbox{A} \\
%               \quarterlinech{选项1}{选项2}{选项3}{选项4}
%               
%               \setmainitem{判断题}{每小题2分,共30分,请在每题后的括号里填写正确的判断结果}
%               \mainitem
%               \question 判断题1 可以将数组名作为左值 \selectedbox{A}
%               \question 判断题2 指针变量可以加减 \selectedbox{A}
%               
%               \setmainitem{简答题}{共30分,每小题分标在小题后,请简明扼要的对每个题目进行解答}
%               \mainitem
%               \question 简答题1 是一道简答题
%               \begin{parts}
%                       \part[3] 第1 小问
%                       \part[5] 第2 小问
%               \end{parts}
%                
%               \setmainitem{论述题}{共30分,每小题分标在小题后,请对下面的题目进行全面解答}
%               \mainitem
%               \question 论述题1 试论数组与指针结合的高级操作
%               \begin{parts}
%                       \part[5] 第1 小问
%                       \part[5] 第2 小问
%               \end{parts} 
%               
%               \setmainitem{计算题}{共30分,每小题分标在小题后,请计算各个题目,并对详细表达计算过程}
%               \mainitem
%               \question 计算题1 计算论数组与指针的运算运算结果
%               \begin{parts}
%                       \part[10] 第1 小问
%                       \part[15] 第2 小问
%               \end{parts} 
%               
%               \setmainitem{改错题}{每小题2分,共30分,请改正每题中的错误}
%               \mainitem
%               \question 改错题1 对数组名进行++运算可指向下一个元素。
%               \question 改错题2 函数可以返回多个值。
%
%               \setmainitem{程序设计题}{共30分,每小题分标在小题后,请分析个题目,并详细表达设计过程}
%               \mainitem
%               \question 程序设计题1 设计程序实现两个数的交换
%               \question 程序设计题2 设计递归程序实现快速排序
%                       
%               \setmainitem{制图题}{共30分,请分析个题目,并详细制图设计过程}
%               \mainitem
%               \question 制图题1 请绘制折半查找的程序流程图
%               
%               \setmainitem{其它题型}{共30分,请分析个题目,并详细制图设计过程}
%               \mainitem
%               \question 其它题型1 设计泛型算法的流程图、程序和测试用例
%               \begin{parts}
%                       \part[10] 第1 流程图
%                       \part[6] 第2 程序
%                       \part[4] 第2 测试用例
%               \end{parts}     
%                                                       
%       \end{questions}
% \end{verbatim}
% 关于题目的更多内容,请参考\ref{sec:Environment}。
%
% \section{使用帮助}
%
% \subsection{选项}
% \label{sec:Options}
%

% 宏包的选项用于改变一些缺省的设置。虽然缺省的设置基本能过满足一般用户的
% 使用需要,但用户也可以根据自己的情况,使用这些选项。
%
% \begin{description}
% \item[answers]     在每一个问题后附上答案。
% \item[sealedpaper]  是否为密封试卷(缺省为普通试卷)。
% \item[contitemcnt]  是否为各小题题目进行连续编号(缺省为不连续编号)
% \item[noindent]    没有缩进。
% \item[prescorebox]    显示前置评分框(缺省为不显示)。
% \end{description}
%
% \subsection{变量}
% \label{sec:Variable}
% 本宏包是参考\nwsuaf 教务处颁布的试卷规范的开发的,
% 默认的变量值可以在\texttt{nwsuafexam.cfg}中设置,当然
% 你也可以在使用相关命令之前使用以下命令进行更改。\\\\
% \DescribeMacro{\papertitle}
% 设置标题信息。
% \begin{quote}
% |\papertitle|\marg*{TEXT}
% \end{quote}
% \DescribeMacro{\startyear}
% 设置副标题中的学年起始年。
% \begin{quote}     
% |\startyear|\marg*{TEXT}
% \end{quote}
% \DescribeMacro{\closeyear}
% 设置副标题中的学年结束年。
% \begin{quote}     
% |\closeyear|\marg*{TEXT}
% \end{quote}
% \DescribeMacro{\semester}
% 设置副标题中的学期号。
% \begin{quote}     
% |\semester|\marg*{TEXT}
% \end{quote}
% \DescribeMacro{\subject}
% 设置副标题中的课程名称。
% \begin{quote}     
% |\subject|\marg*{TEXT}
% \end{quote}
% \DescribeMacro{\papertype}
% 设置副标题中开闭卷类型。
% \begin{quote}     
% |\papertype|\marg*{TEXT}
% \end{quote}
% \DescribeMacro{\papermaker}
% 设置副标题中的命题教师。
% \begin{quote}     
% |\papermaker|\marg*{TEXT}
% \end{quote}
% \DescribeMacro{\papermsg}
% 设置总分和考试时间信息,\parg{num1}为总分,\parg{num2}为考试时间。
% \begin{quote}
% |\papermsg|\marg*{num1}\marg*{num2}
% \end{quote}
% \DescribeMacro{\setmainitem}
% 设置大题题目名称,小分、总分和题目描述信息,\parg{num1}为题目名称,
% \parg{num2}为小分、总分和题目描述信息。
% \begin{quote}
% |\setmainitem|\marg*{TEXT}\marg*{TEXT}
% \end{quote}
%
% \subsection{环境}
% \label{sec:Environment} 
% 经常使用的环境有\verb"questions"环境、\verb"parts"环境,
% 关于它们的简单示例,请参考\ref{sec:Example},这里做一点补充说明,
% 更详细的介绍,请参考~\exam~文档。\\\\
% 在排版简答题时需要用\verb"\part"命令输入各小问的分值,宏包会自动算出总分并显示在该简答题的第一行。
% 当该道简答题没有小问时,则要用\verb"\question"命令输入该问题的分值。
% 
% \begin{verbatim}
%  ...
%  \shortans
%  \question 这是一道简答题
%  \begin{parts}[
%  \part[3] 第1小问3分。
%  \part[3] 第2小问3分。
%  \part[4] 第3小问4分。
%  \end{parts}
%  \question[12] 这是一道没有小问的简答题,这道题有12分
%  ... 
% \end{verbatim}
% 在\verb"\question"后新建\verb"solution"环境,在其中输入该问题的解答,
% 在不显示答案的情况下,该问题后会预留答题空间。
% \begin{verbatim}
%  ...
%  \shortans
%  \question 这是一道简答题
%  \begin{solution}
%  这是解答,不显示答案的情况下则这个问题后预留答题空间。
%  \end{solution}
%  ... 
% \end{verbatim}
%
% \subsection{常用命令}
% 
% \DescribeMacro{\newpage}
% 每道问题的间距是弹性设置的,你只要在想换页的地方输入\verb"\newpage"命令,
% 则上一页的各问题间距会自动调整到最美观的效果。\\
% \DescribeMacro{\mininotice}
% 在一行内输出精简的考试注意事项。\\
% \DescribeMacro{\printmalol}
% 在当前页为正面时,在左边插入装订线(仅在使用sealedpaper选项时有效)。\\
% \DescribeMacro{\printmalor}
% 在当前页为反面时,在右边插入装订线(仅在使用sealedpaper选项时有效)。\\
%
% \StopEventually{}
%
% \section{源代码说明}
%
% \subsection{选项}
%
%
% 处理~\nwsuafexam~文档类的选项
%
% \begin{macro}{\@sealedpaper}
% 是否为密封卷
%    \begin{macrocode}
%<*cls>
\newif\if@sealedpaper \@sealedpaperfalse
\DeclareOption{sealedpaper}{\@sealedpapertrue}
%</cls>
%    \end{macrocode}
% \end{macro}
%
% \begin{macro}{\@contitemcnt}
% 小题题目是否连续编号
%    \begin{macrocode}
%<*cls>
\newif\if@contitemcnt \@contitemcntfalse
\DeclareOption{contitemcnt}{\@contitemcnttrue}
%</cls>
%    \end{macrocode}
% \end{macro}
%
% 不缩进,缺省为缩进
%    \begin{macrocode}
%<*cls>
\newif\if@noindent \@noindentfalse
\DeclareOption{noindent}{\@noindenttrue}
%</cls>
%    \end{macrocode}
%
% 显示答案的方式,缺省不显示答案
%    \begin{macrocode}
%<cls>\DeclareOption{answers}{\PassOptionsToClass{\CurrentOption}{exam}}
%    \end{macrocode}

% \begin{macro}{\@prescorebox}
% 显示前置计分框,缺省为不显示。
%    \begin{macrocode}
%<*cls>
\newif\if@prescorebox \@prescoreboxfalse
\DeclareOption{prescorebox}{\@prescoreboxtrue}
%</cls>
%    \end{macrocode}
% \end{macro}
%
% 把没有定义的选项传递给缺省的文档类
%    \begin{macrocode}
%<cls>\DeclareOption*{\PassOptionsToClass{\CurrentOption}{exam}}
%    \end{macrocode}
%
% 处理选项
%    \begin{macrocode}
%<cls>\ProcessOptions
%    \end{macrocode}
%
% 装入缺省的文档类
%    \begin{macrocode}
%<cls>\LoadClass[addpoints]{exam}
%    \end{macrocode}
% 
% \subsection{加载宏包}
%
% 导入ctex类
%    \begin{macrocode}
%<*cls>
\RequirePackage[zihao=-4]{ctex}
%</cls>
%    \end{macrocode}
%
% \begin{macro}{\RequirePackage}
% 需要使用的其它宏包
%    \begin{macrocode}
%<*cls>
\RequirePackage{ifpdf,ifxetex}
\RequirePackage{amsmath,amssymb,amsthm,bm,bbding,pifont,dsfont}
\RequirePackage{mathtools}
\RequirePackage{paralist,cases,tabularx}
\RequirePackage{pstricks,pst-plot,xcolor,graphicx}
\RequirePackage{ulem}
%</cls>
%    \end{macrocode}
% 
% 用geometry宏包进行页面设置
%    \begin{macrocode}
%<*cls>
\if@sealedpaper
\marginparwidth = 2cm
\RequirePackage[paper=a4paper,hmargin={3.3cm,3.3cm},vmargin={3.3cm,3.3cm},
marginparsep=0.5cm,hoffset=0cm,voffset=0cm,footnotesep=0.5cm,
headsep=0.5cm,twoside]{geometry}
\else
\RequirePackage[papersize={210mm,297mm},hmargin={3.3cm,3.3cm},vmargin={3.3cm,3.3cm},
marginparsep=0.5cm,hoffset=0cm,voffset=0cm,footnotesep=0.5cm,
headsep=0.5cm]{geometry}
\fi
%</cls>
%    \end{macrocode}
%\end{macro}
%
% \subsection{自定义设置}
%行距、页眉、页脚
%    \begin{macrocode}
%<*cls>
\renewcommand{\baselinestretch}{1.5}
\pagestyle{headandfoot}
\header{}{}{}
\footer{}{\small \quad 第~\thepage~页(共~\numpages~页)}{}
%</cls>
%    \end{macrocode}
%
% 分值显示
%    \begin{macrocode}
%<*cls>
\pointname{分}
\pointformat{\kaishu (\thepoints)}
%</cls>
%    \end{macrocode}
%
% 问题的显示
%    \begin{macrocode}
%<*cls>
\renewcommand{\questionshook}{
  \settowidth{\leftmargin}{22.\hskip\labelsep}
  \if@noindent \setlength\leftmargin{0pt} \fi
}
\renewcommand{\thepartno}{\arabic{partno}}
\renewcommand{\partlabel}{(\thepartno)}
\renewcommand{\partshook}{
  \settowidth{\leftmargin}{(3).\hskip\labelsep}
  \if@noindent \setlength\leftmargin{0pt} \fi
}
%</cls>
%    \end{macrocode}
%
% 解答的显示
%    \begin{macrocode}
%<*cls>
\newif\if@cancelspace \@cancelspacetrue
\renewcommand{\solutiontitle}{\noindent 解:\noindent}
\renewenvironment{solution}%
  {%
    \ifprintanswers
        %\unskip
      \begingroup
      \Solution@Emphasis
      \begin{TheSolution}%
    \else
      \if@cancelspace
        %\unskip
      \else
        \par
        \penalty 0
        \vfill%
        \if@prescorebox \if@sufscorebox \sufscorebox \fi \fi
      \fi
      \setbox\z@\vbox\bgroup
    \fi
  }{%
    \ifprintanswers
      \end{TheSolution}%
      \endgroup
    \else
      \egroup
    \fi
  }%
%</cls>
%    \end{macrocode}
%
% \subsection{新的命令和环境}
%
% \begin{macro}{\printmlor}
% \begin{macro}{\printmlol}
% 装订线
%    \begin{macrocode}
%<*cls>
\if@sealedpaper
\newsavebox{\gutterl}
\sbox{\gutterl}{
\begin{minipage}{0.7\paperheight}
\begin{center}
  \heiti 学生院(系):\uline{\hspace{15ex}} \quad
  学号:\uline{\hspace{15ex}} \quad
  姓名:\uline{\hspace{15ex}} \quad
  专业班级:\uline{\hspace{15ex}} \quad \\
  \vspace{3ex}
  \dotfill 装 \dotfill 订 \dotfill 线 \dotfill
\end{center}
\end{minipage}
}
\newsavebox{\gutterr}
\sbox{\gutterr}{
\begin{minipage}{0.7\paperheight}
\begin{center}
\heiti \hfill 请 \hfill 不 \hfill 要 \hfill 在 \hfill
 装 \hfill 订 \hfill 线 \hfill 内 \hfill 答 \hfill 题 \hfill \\
\vspace{3ex}
\dotfill 装 \dotfill 订 \dotfill 线 \dotfill
\end{center}
\end{minipage}
}
\newcommand{\printmlol}{
\marginpar{\rotatebox{90}{\usebox{\gutterl}}}
}
\newcommand{\printmlor}{
\marginpar{\rotatebox{-90}{\usebox{\gutterr}}}
}
\reversemarginpar
\fi
%</cls>
%    \end{macrocode}
% \end{macro}
% \end{macro}
%
% \begin{macro}{\papertitle}
% \begin{macro}{\startyear}
% \begin{macro}{\closeyear}
% \begin{macro}{\semester}
% \begin{macro}{\subject}
% \begin{macro}{\papertype}
% \begin{macro}{\papermaker}
% 标题
%    \begin{macrocode}
%<*cls>
\newcommand\papertitle[1]{\def\@papertitle{#1}}
\newcommand\startyear[1]{\def\@startyear{#1}}
\newcommand\closeyear[1]{\def\@closeyear{#1}}
\newcommand\semester[1]{\def\@semester{#1}}
\newcommand\subject[1]{\def\@subject{#1}}
\newcommand\papertype[1]{\def\@papertype{#1}}
\newcommand\papermaker[1]{\def\@papermaker{#1}}
\renewcommand\maketitle{
  \begin{center}{\heiti \large{\@papertitle}}\end{center}
  \begin{center}{\heiti \large{\@startyear$\sim$\@closeyear 学年第
        \@semester 学期《\@subject》课程\uline{\songti \@papertype}卷}}\end{center}
  \if@sealedpaper
          \begin{center} \heiti 
                \begin{tabularx}{\textwidth}{X<{\centering}X<{\centering}X<{\centering}X<{\centering}X<{\centering}X<{\centering}}
                        命题教师: & \@papermaker & 审题教师:&  & 考试成绩:& \\
                        \cline{2-2}\cline{4-4}\cline{6-6}                       
                \end{tabularx}\\        
                \vspace{0.5em}   
                \hrulefill
          \end{center}
          \marginpar{\rotatebox{90}{\usebox{\gutterl}}}
  \else
          \begin{center} \heiti 
                \begin{tabularx}{\textwidth}{X<{\centering}X<{\centering}X<{\centering}X<{\centering}X<{\centering}X<{\centering}}
                        专业班级: & & 命题教师:& \@papermaker & 审题教师:& \\
                        \cline{2-2}\cline{4-4}\cline{6-6}
                        学生姓名: & & 学~~~~~~号: & & 考试成绩: & \\
                        \cline{2-2}\cline{4-4}\cline{6-6}
                \end{tabularx}\\
                \vspace{0.5em}   
                \hrulefill
          \end{center}
          \fi }
%</cls>
%    \end{macrocode}
% \end{macro}
% \end{macro}
% \end{macro}
% \end{macro}
% \end{macro}
% \end{macro}
% \end{macro}
%
% \begin{macro}{\papermsg}
% \begin{macro}{\mininotice}
% 一行内显示考试时间和考试总分
%    \begin{macrocode}
%<*cls>
\newcommand\papermsg[2]{
  \def\@totalscore{#1}
  \def\@duration{#2}
}
\newcommand\mininotice{
  \begin{center}{
    \kaishu (本试卷满分~\@totalscore~分, 考试时间~\@duration~分钟)}
  \end{center}
}
%</cls>
%    \end{macrocode}
% \end{macro}
% \end{macro}
%
% \begin{macro}{\notice}
% 注意事项
%    \begin{macrocode}
%<*cls>
\newcommand{\notice}{
  \heiti 注意事项: \songti
  \begin{enumerate}
  \item 答卷前, 考生务必将院系、学号、姓名、专业班级等填写清楚.
  \item 本试卷共~\numquestions{}~道试题, 满分~\@totalscore~分,考试时间~\@duration~分钟.
  \end{enumerate}
}
%</cls>
%    \end{macrocode}
%\end{macro}
%
% \begin{macro}{\prescorebox}
% 前评分框
%    \begin{macrocode}
%<*cls>
\newlength\@boxwidth
\setlength\@boxwidth{0ex}
\if@prescorebox \setlength\@boxwidth{18ex} \fi
\newcommand\scorebox{
  \heiti
  \begin{minipage}{\@boxwidth}
  \begin{tabular}{|c|c|}
  \hline
  得分 & 评卷人\\
  \hline
      &       \\
  \hline
  \end{tabular}
  \end{minipage}
}
%</cls>
%    \end{macrocode}
% \end{macro}
%
% \begin{macro}{\sufscorebox}
% 后评分框
%    \begin{macrocode}
%<*cls>
\newcommand{\sufscorebox}{
  \hfill
  \begin{tabular}{|l|l|}
    \hline
    得分 & \hspace*{1.5cm}\\
    \hline
  \end{tabular}
  \bigskip
}
%</cls>
%    \end{macrocode}
% \end{macro}
% 
% \begin{macro}{\selectedbox}
% 选择题答题框
%    \begin{macrocode}
%<*cls>
\newcommand{\selectedbox}[1]{
  \hfill
  \ifprintanswers
    (\uline{~#1~})
  \else
    (\uline{~\hspace{1cm}~})
  \fi
}
%</cls>
%    \end{macrocode}
% \end{macro} 
%
% 一些参数及变量
%    \begin{macrocode}
%<*cfg>
\def\@papertitle{西北农林科技大学本科课程考试试卷}
\def\@startyear{2015}
\def\@closeyear{2016}
\def\@semester{1}
\def\@subject{C~语言程序设计}
\def\@papertype{闭}
\def\@papermaker{耿楠}
\def\@totalscore{100}
\def\@duration{120}

%</cfg>
%<*cls>
\newcounter{@mainitemcounter}
\newif\if@sufscorebox \@sufscoreboxfalse
%</cls>
%    \end{macrocode}
%
% 归一化题型提示语
% \begin{macro}{\setmainitem}
% \begin{macro}{\mainitem}
%    \begin{macrocode}
%<*cls>
\newcommand\setmainitem[2]{
  \def\@mainitemname{#1}
  \def\@mainitemdescription{#2}
}
\newcommand\mainitem{
  \@sufscoreboxfalse
  \stepcounter{@mainitemcounter}
  \if@contitemcnt \else \setcounter{question}{0} \fi
  \fullwidth{
    \if@prescorebox \scorebox \else \@sufscoreboxtrue \fi
    \begin{minipage}{\textwidth-\@boxwidth}
      \heiti \chinese{@mainitemcounter}. \@mainitemname 
      \kaishu ( \@mainitemdescription )
      \if@sufscorebox \sufscorebox \fi
    \end{minipage}
  }
}
%</cls>
%    \end{macrocode}
% \end{macro}
% \end{macro}
%
% \begin{macro}{\shortblk}
% \begin{macro}{\middleblk}
% \begin{macro}{\longblk}
% 填空题横线
%    \begin{macrocode}
%<*cls>
\newcommand{\shortblk}[1]{
  \ifprintanswers
    \uline{~#1~}
  \else
    \uline{~\hspace{1cm}~}
  \fi}
\newcommand{\middleblk}[1]{
  \ifprintanswers
    \uline{~#1~}
  \else
    \uline{~\hspace{2cm}~}
  \fi}
\newcommand{\longblk}[1]{
  \ifprintanswers
    \uline{~#1~}
  \else
    \uline{~\hspace{3cm}~}
  \fi}
%</cls>
%    \end{macrocode}
% \end{macro}
% \end{macro}
% \end{macro}
%
% \begin{macro}{\singlelinech}
% \begin{macro}{\doublelinesch}
% \begin{macro}{\fourlinesch}
% 选择题的选项
%    \begin{macrocode}
%<*cls>
\newcommand{\singlelinech}[4]{
\renewcommand\arraystretch{1.4}
\begin{tabularx}{\linewidth}{XXXX}
\setlength\tabcolsep{0pt}
(A) #1 & (B) #2 & (C) #3 & (D) #4 \\
\end{tabularx}
\unskip \unskip}
\newcommand{\doublelinesch}[4]{
\renewcommand\arraystretch{1.4}
\begin{tabularx}{\linewidth}{XX}
\setlength\tabcolsep{0pt}
(A) #1 & (B) #2 \\
(C) #3 & (D) #4
\end{tabularx}
\unskip \unskip}
\newcommand{\fourlinesch}[4]{
\renewcommand\arraystretch{1.4}
\begin{tabularx}{\linewidth}{X}
\setlength\tabcolsep{0pt}
(A) #1 \\
(B) #2 \\
(C) #3 \\
(D) #4 \\
\end{tabularx}
\unskip \unskip}
%</cls>
%    \end{macrocode}
% \end{macro}
% \end{macro}
% \end{macro}
%
% 数学运算符号、单位
%    \begin{macrocode}
%<*cls>
\delimitershortfall-1sp
\newcommand\abs[1]{\left|#1\right|}
\renewcommand{\geq}{\geqslant}
\renewcommand{\ge}{\geqslant}
\renewcommand{\leq}{\leqslant}
\renewcommand{\le}{\leqslant}
%<\cls>
%    \end{macrocode}
%
% 读取配置文件
%    \begin{macrocode}
%<*cls>
\AtEndOfPackage{\makeatletter%%
%% This is file `nwsuafexam.cfg',
%% generated with the docstrip utility.
%%
%% The original source files were:
%%
%% nwsuafexam.dtx  (with options: `cfg')
%% 
%% This is a generated file.
%% 
%% Copyright (C) 2014 by Geng Nan <nangeng@gq.com>
%% 
%% This file was generated from file(s) of the nwsuafexam package project.
%% ----------------------------------------------------------------------
%% 
%% It may be distributed and/or modified under the
%% conditions of the LaTeX Project Public License, either version 1.2
%% of this license or (at your option) any later version.
%% The latest version of this license is in
%%    http://www.latex-project.org/lppl.txt
%% and version 1.2 or later is part of all distributions of LaTeX
%% version 1999/12/01 or later.
%% 
%% This is the configuration file of the nwsuafexam package with LaTeX2e.
%% 
\ProvidesFile{nwsuafexam.cfg}
  [2016/01/04 v0.1 nwsuafexam
   configuration file]


\def\@papertitle{西北农林科技大学本科课程考试试卷}
\def\@startyear{2015}
\def\@closeyear{2016}
\def\@semester{1}
\def\@subject{C~语言程序设计}
\def\@papertype{闭}
\def\@papermaker{耿楠}
\def\@totalscore{100}
\def\@duration{120}

\endinput
%%
%% End of file `nwsuafexam.cfg'.
\makeatother}
%</cls>
%    \end{macrocode}
%
%

% \Finale
%
% \setcounter{IndexColumns}{2}
% \IndexPrologue{\section*{索引}}
%
% \GlossaryPrologue{\section*{版本更新}}
%
% \PrintIndex \PrintChanges
\endinput

\EnableCrossrefs

% \endinput
% Local Variables:
% mode: doctex
% TeX-master: t
% End:
